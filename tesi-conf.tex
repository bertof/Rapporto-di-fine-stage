%**************************************************************
% file contenente le impostazioni della tesi
%**************************************************************

%**************************************************************
% Impostazioni KOMA Script
%**************************************************************

% Margini
\KOMAoptions{DIV=14}
\addtokomafont{disposition}{\rmfamily}
\KOMAoptions{fontsize=11pt}


%**************************************************************
% Comandi custom
%**************************************************************

% Autore
\newcommand{\myName}{Filippo Berto}                                    
\newcommand{\myTitle}{Sviluppo di una piattaforma di video streaming per l'assistenza remota tramite dispositivi wearable}

% Tipo di tesi                   
\newcommand{\myDegree}{Tesi di laurea triennale}

% Università             
\newcommand{\myUni}{Università degli Studi di Padova}

% Facoltà       
\newcommand{\myFaculty}{Corso di Laurea in Informatica}

% Dipartimento
\newcommand{\myDepartment}{Dipartimento di Matematica `Tullio Levi-Civita`}

% Titolo del relatore
\newcommand{\profTitle}{Prof.}

% Relatore
\newcommand{\myProf}{Tullio Vardanega}

% Luogo
\newcommand{\myLocation}{Padova}

% Anno accademico
\newcommand{\myAA}{2016--2017}

% Data discussione
\newcommand{\myTime}{Settembre 2017} %TODO: check solo mese

% Nome azienda
\newcommand{\nomeAzienda}{Vision Lab Apps}

% Nome azienda commerciale
\newcommand{\nomeAziendaComm}{Vision Lab Apps Srl}

%**************************************************************
% Parole in Inglese
%**************************************************************

% Comando per parole in inglese: aggiunge emph e imposta la lingua
\newcommand{\engl}[1]{\emph{\foreignlanguage{english}{#1}}}

%**************************************************************
% Impostazioni di biblatex %TODO: check
%**************************************************************

%\bibliography{bibliografia} % database di biblatex 

%\defbibheading{bibliography} {
%    \cleardoublepage{}
%    \phantomsection{}
%    \addcontentsline{toc}{chapter}{\bibname}
%    \chapter*{\bibname\markboth{\bibname}{\bibname}}
%}

%\setlength\bibitemsep{1.5\itemsep} % spazio tra entry

%\DeclareBibliographyCategory{opere}
%\DeclareBibliographyCategory{web}

%\addtocategory{opere}{womak:lean-thinking}
%\addtocategory{web}{site:agile-manifesto}

%\defbibheading{opere}{\section*{Riferimenti bibliografici}}
%\defbibheading{web}{\section*{Siti Web consultati}}


%**************************************************************
% Impostazioni di glossaries
%**************************************************************
%**************************************************************
% Glossario
%**************************************************************
\newglossaryentry{AWS}
{
    name={AWS},
    description={Amazon Web Services (AWS) è una collezione di servizi di cloud computing on demand offerta da Amazon},
    first={Amazon Web Services (AWS)},
    long={Amazon Web Services}
}

\newglossaryentry{VPS}
{
    name={VPS},
    description={Un Virtual Private Server (VPS) è un'istanza di un sistema che viene eseguito in un ambiente virtuale},
    first={Virtual Private Server (VPS)},
    long={Virtual Private Server}
}

\newglossaryentry{IoT}
{
    name={IoT},
    description={Per Internet of Things (IoT) ci si riferisce all'estensione di Internet agli oggetti comuni, che diventano intelligenti e comunicano dati su se stessi e sul mondo che li circonda e allo stesso tempo accedere ad informazioni altrove nella rete},
    first={IoT},
    long={Internet of Things}
}

\newglossaryentry{wearable}
{
    name={Wearable},
    text={wearable},
    description={Si dice wearable un dispositivo elettronico indossabile on impiantabile. In generale questi dispositivi offrono delle funzionalità di notifica legate agli smartphone oppure contengono sensori per la rilevazione di attività fisica e sono un esempio di dispositivo \gls{IoT}}
}

\newglossaryentry{SEO}
{
    name={SEO},
    description={Si definisce Search Engine Optimization (SEO) l'attività di ottimizzazione dei contenuti di una pagina web per l'indicizzazione da parte dei motori di ricerca},
    first={SEO},
    long={Search Engine Optimization}
}

\newglossaryentry{CVS}
{
    name={CVS},
    description={É detto Concurrent Versioning System (CVS) un software che implementa un sistema di controllo di versione. Il sistema mantiene organizzati i cambiamenti fatti ad un certo numero di file e permette a molti sviluppatori di collaborare accedendo alle stesse risorse},
    first={CVS},
    long={Concurrent Versioning System}
} % database di termini
\makeglossaries{}