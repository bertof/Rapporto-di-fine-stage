\documentclass[
   10pt,                               % corpo del font
   a4paper,                            % formato foglio (default A4)
   DIV=calc,                           % margini
   english,
   italian
   ]{scrreprt}

\usepackage[T1]{fontenc}
\usepackage[utf8]{inputenc}

%**************************************************************
% Importazione package
%**************************************************************
\usepackage[english, italian]{babel}   % per scrivere in italiano e in inglese;
                                       % l'ultima lingua è predefinita

\usepackage{tocbibind}                 % pacchetto per la gestione degli indici

\usepackage{booktabs}                  % tabelle
\usepackage{tabularx}                  % tabelle di larghezza prefissata\usepackage{longtable}                 % tabelle su più pagine\usepackage{ltxtable}                  % tabelle su più pagine e adattabili in larghezza

\usepackage{csquotes}                  % gestisce automaticamente i caratteri (")

\usepackage{epigraph}					   % per epigrafi

\usepackage{eurosym}                   % simbolo dell'euro

\usepackage{textcomp}                   % supporto ai simboli, come °

\usepackage{graphicx}                  % immagini

\usepackage[space]{grffile}            % immagini con spazi nel path

\usepackage{hyperref}                  % collegamenti ipertestuali

\usepackage{listings}                  % codici

\usepackage[dvipsnames]{xcolor}        % colori

\usepackage{booktabs}                  % tabelle\usepackage{tabularx}                  % tabelle di larghezza prefissata\usepackage{longtable}                 % tabelle su più pagine\usepackage{ltxtable}                  % tabelle su più pagine e adattabili in larghezza

\usepackage{float}                     % utile per forzare figure dopo del testo con [H]

\usepackage{svg}                       % package per l'utilizzo di immagini in svg

\usepackage{caption}                   % package per gestire meglio le captions

\usepackage[
   toc,                                % glossario nella table of contents
   acronym                             % glossario di tipo acronym
   ]{glossaries}                       % glossario
\makeglossaries{}

%\usepackage[backend=biber,style=verbose-ibid,hyperref,backref]{biblatex}
                                       % eccellente pacchetto per la bibliografia;                                       % produce uno stile di citazione autore-anno;                                       % lo stile "numeric-comp" produce riferimenti numerici
                                       % per includerlo nel documento bisogna:
                                       % 1. compilare una prima volta tesi.tex;
                                       % 2. eseguire: biber tesi
                                       % 3. compilare ancora tesi.tex.

%**************************************************************
% file contenente le impostazioni della tesi
%**************************************************************

%**************************************************************
% Impostazioni KOMA Script
%**************************************************************

% Margini
\KOMAoptions{DIV=14}
\addtokomafont{disposition}{\rmfamily}
\KOMAoptions{fontsize=11pt}


%**************************************************************
% Comandi custom
%**************************************************************

% Autore
\newcommand{\myName}{Filippo Berto}                                    
\newcommand{\myTitle}{Sviluppo di una piattaforma di video \engl{streaming} per l'assistenza remota tramite dispositivi wearable}

% Tipo di tesi                   
\newcommand{\myDegree}{Tesi di laurea triennale}

% Università             
\newcommand{\myUni}{Università degli Studi di Padova}

% Facoltà       
\newcommand{\myFaculty}{Corso di Laurea in Informatica}

% Dipartimento
\newcommand{\myDepartment}{Dipartimento di Matematica `Tullio Levi-Civita`}

% Titolo del relatore
\newcommand{\profTitle}{Prof.}

% Relatore
\newcommand{\myProf}{Tullio Vardanega}

% Luogo
\newcommand{\myLocation}{Padova}

% Anno accademico
\newcommand{\myAA}{2016--2017}

% Data discussione
\newcommand{\myTime}{Settembre 2017}

% Nome azienda
\newcommand{\nomeAzienda}{Vision Lab Apps}

% Nome azienda commerciale
\newcommand{\nomeAziendaComm}{Vision Lab Apps S.r.l.}

%**************************************************************
% Parole in Inglese
%**************************************************************

% Comando per parole in inglese: aggiunge emph e imposta la lingua
\newcommand{\engl}[1]{\emph{\foreignlanguage{english}{#1}}}

%**************************************************************
% Impostazioni di glossaries
%**************************************************************
%**************************************************************
% Glossario
%**************************************************************
\newglossaryentry{AWS}
{
    name={AWS},
    description={Amazon Web Services (AWS) è una collezione di servizi di cloud computing on-demand offerta da Amazon},
    first={Amazon Web Services (AWS)},
    long={Amazon Web Services}
}

\newglossaryentry{VPS}
{
    name={VPS},
    description={Un Virtual Private Server (VPS) è un'istanza di un sistema che viene eseguito in un ambiente virtuale},
    first={Virtual Private Server (VPS)},
    long={Virtual Private Server}
}

\newglossaryentry{IoT}
{
    name={IoT},
    description={Per Internet of Things (IoT) ci si riferisce all'estensione di Internet agli oggetti comuni, che diventano intelligenti e comunicano dati su se stessi e sul mondo che li circonda e allo stesso tempo accedere ad informazioni altrove nella rete},
    first={IoT},
    long={Internet of Things}
}

\newglossaryentry{wearable}
{
    name={Wearable},
    text={wearable},
    description={Si dice wearable un dispositivo elettronico indossabile on impiantabile. In generale questi dispositivi offrono delle funzionalità di notifica legate agli smartphone oppure contengono sensori per la rilevazione di attività fisica e sono un esempio di dispositivo \gls{IoT}}
}

\newglossaryentry{SEO}
{
    name={SEO},
    description={Si definisce Search Engine Optimization (SEO) l'attività di ottimizzazione dei contenuti di una pagina web per l'indicizzazione da parte dei motori di ricerca},
    first={SEO},
    long={Search Engine Optimization}
}

\newglossaryentry{CVS}
{
    name={CVS},
    description={É detto Concurrent Versioning System (CVS) un software che implementa un sistema di controllo di versione. Il sistema mantiene organizzati i cambiamenti fatti a un certo numero di file e permette a molti sviluppatori di collaborare accedendo alle stesse risorse},
    first={CVS},
    long={Concurrent Versioning System}
}

\newglossaryentry{ubiquitous computing}
{
    name={Ubiquitous computing},
    description={L'ubiquitous computing è un nuovo modello di interfaccia uomo macchina, secondo il quale ogni persona, nelle sue azioni quotidiane, può entrare in contatto con un enorme numero di dispositivi elettronici, più o meno specializzati, che comunicano tra loro e possono collaborare a uno scopo. Si differenzia dal precedente modello uomo-macchina per la completa integrazione dell'elaborazione delle informazioni all'interno del singolo dispositivo, senza dipendere da un nodo computazionale esterno},
    text={ubiquitous computing}
}

\newglossaryentry{project manager}
{
    name={Project manager},
    description={Il \engl{project manager} di un progetto è il responsabile dell'organizzazione dei processi e della loro pianificazione all'interno di esso},
    text={project manager}
}

\newglossaryentry{real-time}
{
    name={Real-time},
    description={In informatica un sistema real-time è un sistema in cui la correttezza del risultato delle sue computazioni non solo dipende della correttezza logica, ma anche dal tempo impiegato per raggiungerlo. Un sistema real-time deve poter garantire che il tempo di computazione non superi un certo limite superiore, deciso in progettazione},
    text={real-time}
}

\newglossaryentry{Docker}
{
    name={Docker},
    description={Docker è un software open-source per la virtualizzazione di sistemi operativi in ``container'' isolati e controllati. Il metodo utilizzato da Docker sfrutta il sistema di isolamento delle risorse del kernel Linux, permettendo la coesistenza di più container sulla stessa macchina e limitando gli sprechi di risorse collegati all'utilizzo di una macchina virtuale completa}
}

\newglossaryentry{CMS}{
    name={Content Management System (CMS)},
    text={CMS},
    first={Content Management System (CMS)},
    description={Un Content Management System (CMS) è un software di supporto alla creazione, modifica e gestione di contenuti digitali}
}

\newglossaryentry{CAD}{
    name={Computer-Aided Design (CAD)},
    text={CAD},
    description={Computer-Aided Design (CAD) indica un software volto all'utilizzo di tecnologie per la computer grafica per supportare l'attività di progettazione di modelli, soprattutto 3D}
}

\newglossaryentry{TAC}{
    name={Tomografia Assiale Computerizzata (TAC)},
    text={TAC},
    description={In medicina la Tomografia Assiale Computerizzata (TAC) è una metodica diagnostica per immagini che consente di riprodurre sezioni o strati ed effettuare elaborazioni tridimensionali dei dati ottenuti}
}

\newglossaryentry{thin client}{
    name={Thin client},
    text={thin client},
    description={Un \engl{thin client} è un \engl{client} leggero pensato per connettersi a un server remoto che esegue tutte le le operazioni sensibili. Si contrappone al convenzionale \engl{fat client} nel quale è il client stesso ad eseguire la maggior parte delle operazioni e può comunicare parte dei dati ad altri dispositivi}
}

\newglossaryentry{Gantt}
{
    name={Gantt},
    description={Un diagramma di Gantt è un diagramma a barre pensato per mostrare su una scala temporale le attività di un processo, le risorse che occupano, il tempo impiegato e le dipendenze di ciascuna. Questo diagramma è molto utile per stimare i tempi di sviluppo di un prodotto e fissare milestone e scadenze adeguate}
}

\newglossaryentry{PERT}
{
    name={Program Evaluation and Review Technique (PERT)},
    description={Program Evaluation and Review Technique (PERT). è uno strumento utilizzato per la gestione di un progetto pensato per analizzarne e rappresentarne i task necessari al suo completamento},
    text={PERT}
}

\newglossaryentry{VHS}
{
    name={VHS},
    text={VHS},
    description={descrizione VHS}%TODO: descrizione VHS
}

\newglossaryentry{DAB}
{
    name={Digital Audio Broadcasting (DAB)},
    text={DAB},
    description={descrizione DAB}%TODO: descrizione DAB
} % database di termini
\makeglossaries{}

\begin{document}

%**************************************************************
% Parte iniziale
%**************************************************************

\include{inizio/frontespizio}
\include{inizio/colophon}
%**************************************************************
% Citazione
%**************************************************************
\setlength\epigraphwidth{.8\textwidth}
\setlength\epigraphrule{0pt}

\begin{flushright}
   \large{\textit{Ringrazio il mio tutor aziendale,\linebreak Guido, e l'azienda \nomeAzienda{};}}
\end{flushright}

\begin{flushright}
   \large{\textit{Ringrazio il prof. Vardanega;}}
\end{flushright}

\begin{flushright}
   \large{\textit{Ringrazio i miei genitori,\linebreak per avermi sempre supportato;}}
\end{flushright}

\begin{flushright}
   \large{\textit{Ringrazio la mia fidanzata Chiara,\linebreak per essere sempre al fianco.}}
\end{flushright}

\vspace*{\fill}

\epigraph{\textit{``The first precept was never to accept a thing as true until I knew it as such without a single doubt.''}}{René Descartes} %TODO: citazione + dedica a fondo pagina
%TODO: pagina bianca
%TODO: sommario
%TODO: Ringraziamenti
%**************************************************************
% Indici
%**************************************************************
\cleardoublepage{}
\setcounter{tocdepth}{2}
\tableofcontents
%\markboth{\contentsname}{\contentsname} 
\clearpage
\listoffigures
%\listoftables
\cleardoublepage{}

%**************************************************************
% Contenuti
%**************************************************************

%**************************************************************
% Capitolo 1 - L'azienda
%**************************************************************
\chapter{L'azienda\label{cap:lazienda}}

\section{I servizi}
%TODO: Sezione i servizi dell'azienda

\section{Come lavora}
%TODO: Sezione come lavora

\section{Che tecnologie utilizzate}
L'azienda fa uso di un gran numero di tecnologie durante le proprie attività, di seguito analizzerò le più utilizzate.

\subsection{Rackspace}
\begin{figure}[htbp]
   \begin{center}
      \includegraphics[width=6cm,height=6cm,keepaspectratio]{immagini/rackspace-logo}
   \end{center}
   \caption{Logo di Rackspace}
\end{figure}
Rackspace è un cloud provider che offre servizi di managed cloud computing, basati su VPS e altri servizi cloud, come AWS, Microsoft Azure e OpenStack. Questo tipo di servizio permette di gestire facilmente servizi cloud utilizzati, mantenendo il pieno controllo di costi e infrastrutture, senza la necessità di conoscere a fondo ogni componente utilizzato.
\nomeAzienda{} usa Rackspace come hosting provider nel caso di progetti complessi, quando è necessaria una completa gestione delle risorse.

\subsection{Firebase}
\begin{figure}[htbp]
   \begin{center}
      \includegraphics[width=6cm,height=6cm,keepaspectratio]{immagini/firebase-logo}
   \end{center}
   \caption{Logo di Firebase}
\end{figure}
Firebase è una piattaforma di sviluppo per applicazioni Web e mobile, parte di Google Cloud Platform; fornisce servizi di scambio di messaggi e basi di dati in tempo reale, spazio di archiviazione, sistemi di autenticazione, web hosting e test automatici per applicazioni Android. La piattaforma fornisce anche un servizio di analisi e profilazione degli utenti e l'integrazione con il sistema di annunci pubblicitari di Google.
\nomeAzienda{} utilizza Firebase quando necessità della creazione di un ambiente di sviluppo completo, veloce e facile da manutenere.

%\pagebreak %TODO: find a better fix for the image

\subsection{Java}
\begin{figure}[htbp]
   \begin{center}
      \includegraphics[width=6cm,height=6cm,keepaspectratio]{immagini/java-logo}      
   \end{center}
   \caption{Logo di Java}
\end{figure}
Java è un linguaggio di programmazione ad alto livello orientato agli oggetti pensato per essere il più possibile indipendente dalla piattaforma sulla quale viene eseguito. Java supera questo ostacolo utilizzano una macchina virtuale, la JVM, che permette di astrarre il sistema sottostante. Il vantaggio di Java sui linguaggi compilati tradizionali è proprio quello di poter essere eseguito su una qualsiasi piattaforma, a patto che esista una JVM per questa. Tra le tecnologie utilizzate da \nomeAzienda{} troviamo Android, fortemente basato su Java, e utilizzato per la creazione di applicazioni per dispositivi mobile. Molti dei progetti passati dell'azienda sono legati ad applicazioni Android, ma \nomeAzienda{} utilizza Java anche nel caso di servizi web ad alto parallelismo.

\subsection{Git e Bitbucket}
\begin{figure}[htbp]
   \centering
   {\includegraphics[width=6cm,height=1.2cm,keepaspectratio]{immagini/git-logo} }
   \qquad
   {\includegraphics[width=6cm,height=1.2cm,keepaspectratio]{immagini/bitbucket-logo} }
   \caption{Loghi di Git e Bitbucket}
\end{figure}
\nomeAzienda{} utilizza Git come CVS per il versionamento del codice: Git è in grado di gestire progetti anche molto complessi in modo efficiente; il suo sistema completamente distribuito permette di conservare copie sicure del prodotto in luoghi separati, garantendone la consistenza. Per facilitare la gestione del codice e automatizzare alcune attività, l'azienda ha scelto di utilizzare Bitbucket come hoster per le proprie repository. Bitbucket integra il servizio di pipeline, che permette di eseguire degli script in ambienti virtualizzati basati su Docker; in questo modo sono stati automatizzati i test di unità e integrazione e i controlli della quality assurance.

\subsection{WordPress}
\begin{figure}[htbp]
   \begin{center}
      \includegraphics[width=6cm,height=6cm,keepaspectratio]{immagini/wordpress-logo}
   \end{center}
   \caption{Logo di WordPress}
\end{figure}
WordPress è una piattaforma editoriale personale; nato per gestire semplici blog, viene utilizzato come piattaforma di sviluppo di siti molto più complessi, sfruttando il sistema a plugin su cui e basato. L'utilizzo di WordPress come base di un sito permette di iniziare a lavorare con un framework riutilizzabile, stabile e aggiornato che gestisce i contenuti e i dati del sito, permettendo allo sviluppatore di concentrarsi sulla loro presentazione all'utente.
\nomeAzienda{} sfrutta WordPress come base dei propri siti anche per rendere la modifica dei contenuti semplice al proprio cliente.

\section{Rapporto con l'innovazione}
%TODO: Rapporto con l'innovazione (si cerca l'innovazione o si lavora su solo quello che serve)
\nomeAzienda{} è alla continua ricerca di nuove tecnologie, che possano migliorare efficienza ed efficacia delle proprie attività. Per ogni progetto gli analisti e i progettisti valutano le possibili tecnologie da utilizzare e scelgono quelle che più si adattano al progetto. L'azienda si impegna per restare sempre aggiornata con le nuove tecnologie disponibili sul mercato, tramite            % Introduzione - L'azienda
%**************************************************************
% Capitolo 2 - PERCHÉ - Scelta dello stage e rapporto con l'azienda
%**************************************************************
\chapter{Scelta dello stage e rapporto con l'azienda}
%TODO: introduzione capitolo

\section{Lo stage per l'azienda}
%TODO: sezione: rapporto gli stage dell'azienda

   \subsection{Necessità dell'azienda}

   %TODO: da rivedere
   \nomeAzienda{} ha deciso anche quest'anno di partecipare all'evento ``StageIt'', organizzato da Confindustria Padova in collaborazione con l'Università degli Studi di Padova, e proporre agli studenti uno tirocinio interno all'azienda. L'impresa è alla ricerca di nuovi neolaureati per arricchire il proprio team di sviluppatori, dato il recente aumento di clienti e la conseguente espansione. \nomeAzienda{} è, in particolare, interessata a studenti che sono appassionati di nuove tecnologie e che hanno interesse ad imparare nuove tecniche e a farne conoscere di nuove all'azienda stessa.

   \subsection{Risultati degli stage precedenti e seguito degli stagisti nell'azienda}

\section{Rapporto con il mio stage e l'azienda}
%TODO: sezione cosa mi interessa, cosa mi piace fare

   \subsection{Ambiti di interesse}

   \subsection{Proposte di stage ricevute}

   \subsection{Scelta dello stage}

   \subsection{Scelta dell'azienda}

\section{Obiettivi dello stage}

   \subsection{Obiettibi obbligatori}

   \subsection{Obiettivi desiderabili}

   \subsection{Vincoli tecnologici}

\section{Pianificazione del lavoro}

   \subsection{Scelte sulla pianificazione}

   \subsection{Strumenti utilizzati}            % Scelta dello stage e rapporto con l'azienda
%**************************************************************
% Capitolo 2 - Descrizione dello stage
%**************************************************************
\chapter{Descrizione dello stage\label{cap:descrizionestage}}
\section{Scopo dello stage}

Lo scopo di questo stage è quello di studiare gli attuali metodi di trasmissione di video, in tempo reale e on demand, e di sviluppare un applicativo server in grado di gestire la ritrasmissione di messaggi tra i client connessi.

\section{Analisi preventiva dei rischi}


\section{Pianificazione}            % Progetto di stage

%\appendix%\input{capitoli/capitolo-A}            % Appendice A

%**************************************************************
% Parte finale
%**************************************************************

\printglossary[toctitle=Glossario]{}
%**************************************************************
% Bibliografia
%**************************************************************

\cleardoublepage{}

\begin{thebibliography}{9}
   
   \bibitem{mobiledesktopusage}
      James Titcomb,
      \textit{Mobile web usage overtakes desktop for first time},\\
      \href{www.telegraph.co.uk/technology/2016/11/01/mobile-web-usage-overtakes-desktop-for-first-time}{www.telegraph.co.uk/technology/2016/11/01/\\mobile-web-usage-overtakes-desktop-for-first-time},\\
      The Telegraph,
      1 November 2016
   
   \end{thebibliography}

\end{document}
