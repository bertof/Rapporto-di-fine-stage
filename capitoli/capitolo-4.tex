%**************************************************************
% Capitolo 4 - Analisi retrospettiva
%**************************************************************
\chapter{Analisi retrospettiva}
In questo capitolo realizzerò una breve analisi retrospettiva dello stage e del progetto, dei risultati ottenuti, del raggiungimento degli obiettivi, e delle conoscenze che mi sono state necessarie o utili.

\section{Obiettivi}
\subsection{Obiettivi personali}
\subsection{Obiettivi dell'azienda}
%TODO: da espadere, palesemente
A seguito ai test e alla validazione finale, è stato confermato dal mio tutor aziendale che tutti gli obiettivi fissati dall'azienda, sia quelli obbligatori, sia quelli desiderabili, sono stati raggiunti.

\section{Bilancio formativo}
Dal mio punto di vista, il tirocinio è stato un'ottima esperienza formativa. Ho avuto la possibilità di espandere le mie conoscenze di seguire un progetto in quasi totale autonomia, un'occasione per accrescere la mia capacità di farmi carico di responsabilità. È stato sicuramente molto importante avere la possibilità di confrontarmi con l'ambiente lavorativo di un'azienda ben avviata e di comprenderne i ritmi e i rapporti con le varie cariche.
\\
Per quanto riguarda le conoscenze acquisite, ho potuto studiare e conoscere più approfonditamente protocolli e formati video; questi ultimi non fanno parte del programma del corso laurea in informatica che mi accingo a concludere e sono stati particolarmente interessanti, data la loro natura prettamente materiale. Altre importanti nozioni che ho appreso durante lo stage riguardano le blockchain, delle quali conoscevo solamente le applicazioni pratiche e non il loro funzionamento più teorico, e le reti peer-to-peer, in particolare la rete BitTorrent, di cui non conoscevo né le specifiche, né le possibilità nell'ambito dello streaming video. Ho anche avuto modo di utilizzare praticamente le conoscenze di programmazione concorrente e di Java, apprese durante i corsi, e di altre tecnologie di supporto, come Git e Docker, migliorando considerevolmente le mie competenze.

\section{Conoscenze desiderabili}
% Cosa sarebbe stato utile conoscere per avere avuto un risultato migliore
% Suggerimenti su cosa si potrebbe cambiare in meglio nel corso di studi
Le nozioni apprese durante questo corso di laurea mi sono state di grande aiuto, non solo come competenze necessario allo svolgimento delle attività all'interno di un'azienda, ma anche come base per un più rapido apprendimento di nuove tecnologie e nuove tecniche, come ho potuto constatare durante il tirocinio all'interno di \nomeAzienda{}. Durante il tirocinio ho avuto necessità di conoscere più approfonditamente alcuni aspetti di programmazione e di reti, in particolare le basi di programmazione funzionale in Java, utilizzate per semplificare la codifica del lavoro in parallelo sul server, e l'uso pratico dei protocolli di rete real-time.
\\
Tra le conoscenze che più mi sono state utili durante lo stage posso sicuramente indicare quelle fornitemi dai corsi di Ingegneria del Software e Programmazione Concorrente e Distribuita: il primo per le competenze necessarie all'organizzazione e alla gestione di un progetto, fondamentali per la buona riuscita di questi, nonché i principali design pattern da utilizzare durante la progettazione dei prodotti. Penso, proprio questi ultimi potrebbero essere più utili se ci fosse la possibilità di applicarli in più di un progetto, durante i corsi, magari durante il corso di Programmazione ad Oggetti oppure tramite progetti opzionali dei corsi del terzo anno, come punti aggiuntivi alla valutazione finale. Altro corso le cui nozioni mi sono state particolarmente utili durante il tirocinio è stato il corso di Programmazione Concorrente e Distribuita; in particolare per lo studio di Java e dell'utilizzo di processi paralleli durante l'esecuzione dei processi, fondamentali nel caso di applicazioni di rete che debbano servire un numero enorme di client nel minor tempo possibile. Anche in questo caso penso che il corso ci prepari in modo più che soddisfacente sia sul lato teorico che sul lato pratico, ma che un progetto opzionale potrebbe aiutare a fissare maggiormente le capacità acquisite.

\section{Futuri sviluppi del progetto di stage}
Durante lo sviluppo della piattaforma di streaming sono nate nuove idee riguardo l'implementazione di altri servizi da aggregare alla piattaforma. Tali ampliamenti porterebbero il prodotto ad una maggiore competitività nel mercato e ad un migliore qualità di servizio ai clienti che lo useranno.
\\
La necessità del prodotto è nata nel momento in cui è stato necessario far comunicare due dispositivi Android, in cui stava venendo testato il client in sviluppo, senza incorrere nei problemi legati ad una comunicazione diretta e aggiungendo funzionalità come la struttura a canali del servizio, che permette anche a più di due client di comunicare assieme.
\\
Attualmente il servizio permette di trasmettere dati tra più dispositivi, ma non esegue alcuna elaborazione dei contenuti. Questo modo di operare è stato scelto per poter avere un servizio di scambio di dati generico altamente scalabile, agnostico sui contenuti dei messaggi. Il passaggio successivo, un a volta reso stabile il client Android, sarà quello di realizzare i tre componenti dell'architettura impegnati, rispettivamente, nella ricezione dei flussi, nella loro transcodifica e successivamente nella loro distribuzione.
\\
Un'ulteriore tecnologia che \nomeAzienda{} vuole integrare nel prodotto è quella della condivisione, tramite rete peer-to-peer, dei dati. Come visto tramite nella mia relazione, la quantità di lavoro spostato sui client, a vantaggio dei server di distribuzione, è tale da rendere molto fruttuoso un investimento nello sviluppo di questa funzionalità in una piattaforma di streaming.

\section{Valutazione personale}
Concludo questo Rapporto di fine Stage con una breve valutazione personale.
\\
Mi ritengo soddisfatto del lavoro svolto. Ho avuto modo di studiare, analizzare, progettare e sviluppare il progetto nella sua interezza secondo i requisiti tecnologici, temporali e qualitativi richiesti. Il servizio realizzato è completo e il suo codice verrà utilizzato dall'azienda e integrato nel client Android; inoltre, gli studi eseguiti hanno portato nuove idee al team di sviluppo e suscitato grande interesse tra il personale.
\\
Posso, infine, essere felice di aver potuto lavorare in un'azienda emergente, che punta molto sull'innovazione e sui giovani, con un gruppo di persone entusiaste ad un argomento interessante e particolarmente sentito negli ultimi tempi.