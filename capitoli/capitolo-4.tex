%**************************************************************
% Capitolo 4 - Analisi retrospettiva
%**************************************************************
\chapter{Analisi retrospettiva}
\section{Obiettivi}
\subsection{Obiettivi personali}
\subsection{Obiettivi dell'azienda}
%TODO: da espadere, palesemente
A seguito ai test e alla validazione finale, è stato confermato dal mio tutor aziendale che tutti gli obiettivi fissati dall'azienda sono stati rispettati.

\section{Bilancio formativo}
Dal mio punto di vista, il tirocinio è stato un'ottima esperienza formativa. Ho avuto la possibilità di espandere le mie conoscenze di seguire un progetto in quasi totale autonomia, un'occasione per accrescere la mia capacità di farmi carico di responsbilità. È stato sicuramente molto importante avere la possibilità di confrontarmi con l'ambiente lavorativo di un'azienda ben avviata e di comprenderne i ritmi e i rapporti con le varie cariche.
\\
Per quanto riguarda le conoscenze acquisite, ho potuto studiare e conoscere più approfonditamente protocolli e formati video; questi ultimi non fanno parte del programma del corso laurea in informatica che mi accingo a concludere e sono stati particolarmente interessanti, data la loro natura prettamente materiale. Altre importanti nozioni che ho appreso durante lo stage riguardano le blockchain, delle quali conoscevo solamente le applicazioni pratiche e non il loro funzionamento più teorico, e le reti peer-to-peer, in particolare la rete BitTorrent, di cui non conoscevo né le specifiche, né le possibilità nell'ambito dello streaming video. Ho anche avuto modo di utilizzare praticamente le conoscenze di programmazione concorrente e di Java, apprese durante i corsi, e di altre tecnologie di supporto, come Git e Docker, migliorando considerevolmente le mie competenze.

\section{Nozioni mancanti}
% Cosa sarebbe stato utile conoscere per avere avuto un risultato migliore
% Suggerimenti su cosa si potrebbe cambiare in meglio nel corso di studi

\section{Futuri sviluppi del progetto di stage}
