%**************************************************************
% Capitolo 4 - Analisi retrospettiva
%**************************************************************
\chapter{Analisi retrospettiva}
\section{Obiettivi}
\subsection{Obiettivi personali}
\subsection{Obiettivi dell'azienda}
%TODO: da espadere, palesemente
A seguito ai test e alla validazione finale, è stato confermato dal mio tutor aziendale che tutti gli obiettivi fissati dall'azienda sono stati rispettati.

\section{Bilancio formativo}
Dal mio punto di vista, il tirocinio è stato un'ottima esperienza formativa. Ho avuto la possibilità di espandere le mie conoscenze di seguire un progetto in quasi totale autonomia, un'occasione per accrescere la mia capacità di farmi carico di responsabilità. È stato sicuramente molto importante avere la possibilità di confrontarmi con l'ambiente lavorativo di un'azienda ben avviata e di comprenderne i ritmi e i rapporti con le varie cariche.
\\
Per quanto riguarda le conoscenze acquisite, ho potuto studiare e conoscere più approfonditamente protocolli e formati video; questi ultimi non fanno parte del programma del corso laurea in informatica che mi accingo a concludere e sono stati particolarmente interessanti, data la loro natura prettamente materiale. Altre importanti nozioni che ho appreso durante lo stage riguardano le blockchain, delle quali conoscevo solamente le applicazioni pratiche e non il loro funzionamento più teorico, e le reti peer-to-peer, in particolare la rete BitTorrent, di cui non conoscevo né le specifiche, né le possibilità nell'ambito dello streaming video. Ho anche avuto modo di utilizzare praticamente le conoscenze di programmazione concorrente e di Java, apprese durante i corsi, e di altre tecnologie di supporto, come Git e Docker, migliorando considerevolmente le mie competenze.

\section{Conoscenze desiderabili}
% Cosa sarebbe stato utile conoscere per avere avuto un risultato migliore
% Suggerimenti su cosa si potrebbe cambiare in meglio nel corso di studi
Le nozioni apprese durante questo corso di laurea mi sono state di grande aiuto, non solo come competenze necessario allo svolgimento delle attività all'interno di un'azienda, ma anche come base per un più rapido apprendimento di nuove tecnologie e nuove tecniche, come ho potuto constatare durante il tirocinio all'interno di \nomeAzienda{}. Durante il tirocinio ho avuto necessità di conoscere più approfonditamente alcuni aspetti di programmazione e di reti, in particolare le basi di programmazione funzionale in Java, utilizzate per semplificare la codifica del lavoro in parallelo sul server, e l'uso pratico dei protocolli di rete real-time.
\\
Tra le conoscenze che più mi sono state utili durante lo stage posso sicuramente indicare quelle fornitemi dai corsi di Ingengeria del Software e Programmazione Concorrente e Distribuita: il primo per le competenze necessarie all'organizzazione e alla gestione di un progetto, fondamentali per la buona riuscita di questi, nonché i principali design patter da utilizzare durante la progettazione dei prodotti. Penso, proprio questi utlitmi potrebbero essere più utili se ci fosse la possibilità di applicarli in più di un progetto, durante i corsi, magari durante il corso di Programmazione ad Oggetti oppure tramite progetti opzionali di Programmazione Concorrente e Distribuita e di Tecnologie Web. Altro corso le cui nozioni mi sono state particolarmente utili durante il tirocinio è stato il corso di P. C. e D., in particolare per lo studio di Java e dell'utilizzo di processi paralleli durante l'esecuzione dei processi. 

\section{Futuri sviluppi del progetto di stage}
