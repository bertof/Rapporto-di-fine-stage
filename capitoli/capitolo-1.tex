%**************************************************************
% Capitolo 1 - L'azienda
%**************************************************************
\chapter{L'azienda\label{cap:lazienda}}

\section{Prodotti e servizi}
%TODO: Sezione i servizi dell'azienda
\nomeAzienda{} 

\section{Come lavora}

   \subsection{Modello di sviluppo}
   \nomeAzienda{} lavora con il modello di sviluppo Agile di tipo Scrum. Questo modello pone una minore rigidità sulla documentazione e sullo sviluppo del prodotto, permettendo modifiche in corso d'opera.
   \\
   Scrum definisce uno sprint come l'unità di misura dello sviluppo di un progetto, un periodo di tempo di lunghezza fissata, generalmente di una settimana.
   Gli sprint sono pianificati tramite una riunione, durante la quale sono definiti gli obiettivi dello sprint. Le attività di ciascun processo vengono divise e spezzettate dallo scrum master, fino ad ottenere task atomici, che vengono organizzati in base alle loro dipendenze ed assegnati ai componenti del team. Questo tipo di divisione garantisce una visione totale dei task all'interno di uno sprint e permette il massimo parallelismo sul loro avanzamento.
   \\
   Ogni giorno il team si ritrova con una breve riunione, detta ``daily scrum'', per controllare lo stato dei task e degli obiettivi. I meeting giornalieri permettono al project manager di avere misure dello stato del progetto con più frequenza, rispetto ad altri modelli di sviluppo, così da intervenire più rapidamente alla necessità di correzioni. 
   Un altro vantaggio del metodo Scrum, e più in generale dei modelli agili, è quello di poter vedere il risultato del proprio lavoro più in fretta rispetto ai metodi tradizionali: i daily scrum servono anche ad incentivare gli sviluppatori e a fornire loro una sensazione di progresso, che, invece, viene persa se i tempi tra un aggiornamento e l'altro si dilatano.
   \\
   Il coordinamento del lavoro viene gestito tramite fogli di calcolo con funzioni automatiche condivisi all'interno del team. Per ogni task è segnalato il livello di avanzamento, che deve essere aggiornato da colui a cui è stato assegnato, riportando il tempo impiegato ed eventuali note.
   \\
   Gli stati in cui un task si può trovare sono i seguenti:
   \begin{itemize}
      \item{\textbf{Analysis}: il task richiede analisi}
      \item{\textbf{Pending}: il task è definito ed è in attesa di essere svolto}
      \item{\textbf{Blocked}: il task è bloccato a causa delle sue dipendenze}
      \item{\textbf{Development}: il task è in svolgimento}
      \item{\textbf{Testing}: lo sviluppo è stato completato, ma il codice sta venendo testato}
      \item{\textbf{Reworking}: lo sviluppo è fallito e sta venendo rieseguito}
      \item{\textbf{Refactoring}: lo sviluppo è completato, ma il codice sta venendo ripulito }
      \item{\textbf{Completed}: lo sviluppo è completato}
      \item{\textbf{Confirmed}: il task è stato validato}
   \end{itemize}
   Il sistema di tracking del tempo impegnato da ciascun task aiuta il project manager a valutare lo stato del progetto, confrontandolo con le stime fatte a preventivo.

   \subsection{Progetti importanti}
   %TODO: espandere sezione progetti importanti
   \paragraph{VisonHealthCare}
   VisionHealthCare è un software prodotto da \nomeAzienda{} in collaborazione con Dedalus Spa\footnote{Sito web: \href{www.dedalus.eu}{www.dedalus.eu}}, società leader nazionale nel software clinico sanitario. L'applicazione, legata a OrmaWeb, suite applicativa web di Dedauls Spa, sfrutta gli occhiali per la realtà aumentata di Google, i Google Glass, per automatizzare e semplificare ogni fase del percorso chirurgico, dalla lista d'attesa alla gestione del blocco operatorio, fino alla produzione del registro operatorio e la redazione della cartella anestesiologica pre e intraoperatoria. L'applicazione permete a chi indossa gli occhiali di registrare, durante l'operazione, note vocali correlate da video e foto, utili alla documentazione dell'operazione e utilizzabili poi anche per attività di insegnamento o di ricerca.

   \subsection{Premi e certificazioni}
   %TODO: sezione premi e certificazioni

\section{Tecnologie utilizzate}
L'azienda fa uso di un gran numero di tecnologie durante le proprie attività, di seguito analizzerò le più utilizzate.

   \subsection{Rackspace}
   \begin{figure}[H]
      \begin{center}
         \includegraphics[width=6cm,height=6cm,keepaspectratio]{immagini/rackspace-logo}
      \end{center}
      \caption{Logo di Rackspace}\label{logorackspace}
   \end{figure}
   Rackspace è un cloud provider che offre servizi di managed cloud computing, basati su VPS e altri servizi cloud, come AWS, Microsoft Azure e OpenStack. Questo tipo di servizio permette di gestire facilmente servizi cloud utilizzati, mantenendo il pieno controllo di costi e infrastrutture, senza la necessità di conoscere a fondo ogni componente utilizzato.
   \nomeAzienda{} usa Rackspace come hosting provider nel caso di progetti complessi, quando è necessaria una completa gestione delle risorse.

   \subsection{Firebase}
   \begin{figure}[H]
      \begin{center}
         \includegraphics[width=6cm,height=6cm,keepaspectratio]{immagini/firebase-logo}
      \end{center}
      \caption{Logo di Firebase}\label{logofirebase}
   \end{figure}
   Firebase è una piattaforma di sviluppo per applicazioni Web e mobile, parte di Google Cloud Platform; fornisce servizi di scambio di messaggi e basi di dati in tempo reale, spazio di archiviazione, sistemi di autenticazione, web hosting e test automatici per applicazioni Android. La piattaforma fornisce anche un servizio di analisi e profilazione degli utenti e l'integrazione con il sistema di annunci pubblicitari di Google.
   \nomeAzienda{} utilizza Firebase quando necessità della creazione di un ambiente di sviluppo completo, veloce e facile da manutenere.

   %\pagebreak %TODO: find a better fix for the image

   \subsection{Java}
   \begin{figure}[H]
      \begin{center}
         \includegraphics[width=6cm,height=6cm,keepaspectratio]{immagini/java-logo}      
      \end{center}
      \caption{Logo di Java}\label{logojava}
   \end{figure}
   Java è un linguaggio di programmazione ad alto livello orientato agli oggetti pensato per essere il più possibile indipendente dalla piattaforma sulla quale viene eseguito. Java supera questo ostacolo utilizzano una macchina virtuale, la JVM, che permette di astrarre il sistema sottostante. Il vantaggio di Java sui linguaggi compilati tradizionali è proprio quello di poter essere eseguito su una qualsiasi piattaforma, a patto che esista una JVM per questa. Tra le tecnologie utilizzate da \nomeAzienda{} troviamo Android, fortemente basato su Java, e utilizzato per la creazione di applicazioni per dispositivi mobile. Molti dei progetti passati dell'azienda sono legati ad applicazioni Android, ma \nomeAzienda{} utilizza Java anche nel caso di servizi web ad alto parallelismo.

   \subsection{Git e Bitbucket}
   \begin{figure}[H]
      \centering
      {\includegraphics[height=1.5cm,keepaspectratio]{immagini/git-logo} }
      \qquad
      {\includegraphics[height=1.5cm,keepaspectratio]{immagini/bitbucket-logo} }
      \caption{Loghi di Git e Bitbucket}\label{loghigitbitbucket}
   \end{figure}
   \nomeAzienda{} utilizza Git come CVS per il versionamento del codice: Git è in grado di gestire progetti anche molto complessi in modo efficiente; il suo sistema completamente distribuito permette di conservare copie sicure del prodotto in luoghi separati, garantendone la consistenza. Per facilitare la gestione del codice e automatizzare alcune attività, l'azienda ha scelto di utilizzare Bitbucket come hoster per le proprie repository. Bitbucket integra il servizio di pipeline, che permette di eseguire degli script in ambienti virtualizzati basati su Docker; in questo modo sono stati automatizzati i test di unità e integrazione e i controlli della quality assurance.

   \subsection{WordPress}
   \begin{figure}[H]
      \begin{center}
         \includegraphics[width=6cm,height=6cm,keepaspectratio]{immagini/wordpress-logo}
      \end{center}
      \caption{Logo di WordPress}\label{logowordpress}
   \end{figure}
   WordPress è una piattaforma editoriale personale; nato per gestire semplici blog, viene utilizzato come piattaforma di sviluppo di siti molto più complessi, sfruttando il sistema a plugin su cui e basato. L'utilizzo di WordPress come base di un sito permette di iniziare a lavorare con un framework riutilizzabile, stabile e aggiornato che gestisce i contenuti e i dati del sito, permettendo allo sviluppatore di concentrarsi sulla loro presentazione all'utente.
   \nomeAzienda{} sfrutta WordPress come base dei propri siti anche per rendere la modifica dei contenuti semplice al proprio cliente.

\section{Corsa all'innovazione}
%TODO: Rapporto con l'innovazione (si cerca l'innovazione o si lavora su solo quello che serve)
\nomeAzienda{} è da sempre alla continua ricerca di nuove tecnologie da conoscere ed integrare nei propri prodotti, anche in campi sperimentali, come i dispositivi wearable, IoT e la realtà aumentata. Proprio questi ultimi hanno dato origine ad alcuni dei progetti più all'avanguardia dell'azienda e l'hanno spinta all'acquisizione di personale dedito alla sperimentazione di nuove soluzioni.
\\
Un'ulteriore necessità di innovazione deriva dal settore nel quale \nomeAzienda{} si propone: il mercato è in rapida crescita e questo impone un continuo aggiornamento delle conoscenze e delle tecniche per mantenere i propri prodotti validi e restare al passo con i competitor.
\\
Testimonianza di questo continuo aggiornamento è la migrazione verso uno sviluppo cloud based di molti dei prodotti dell'azienda, che ha portato ad una riduzione dei costi di manutenzione e ad un maggiore controllo sulla disponibilità dei servizi.
\\
La proposta di nuove tecnologie è libera all'interno dell'azienda e, se ritenute utili per progetti futuri, viene predisposto un piccolo progetto di prova. In questo modo si possono avere dati concreti sui vantaggi e gli svantaggi che possono offrire.