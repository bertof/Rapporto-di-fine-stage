%**************************************************************
% Capitolo 2 - COSA-COME - Il prodotto dello stage
%**************************************************************
\chapter{Il prodotto\label{cap:ilprodotto}}
\section{Acquisizione delle conoscenze settoriali}
   \subsection{Codifica video}
      %Definizione di codifica video
      %Definizione di codec
      \subsubsection{Storia della codifica video}
      %primi standard di codifica video
      %cambiamenti
      \subsubsection{Formati comunemente usati}
         \paragraph{M-JPEG}
         Motion JPEG  è un codec video che comprime ogni frame del video originale in una immagine in formato JPEG, un comune formato di compressione con perdita per immagini digitali. Originariamente creato per applicazioni multimediali, M-JPEG è attualmente utilizzado da dispositivi di videoregistazione come camere digitali, IP cam e webcam.
         \\
         Un grande vantaggio dell'utilizzo di M-JPEG per la codifica video, che ha spinto la sua diffusione sulla maggior parte delle prime fotocamere digitali non professionali, è quello della bassa potenza di calcolo neessaria al suo utilizzo: è sufficiente una modifica software all'encoder hardware per JPEG per poter creare un codec M-JPEG.\@ Inoltre, permette la lettura non lineare del file, utile nel caso di sistemi di video editing. Inoltre l'encoding intraframe che sfrutta limita l'informazione contenuta in ciascun frame M-JPEG a quella di un frame del video originale, permettendo di cambiare velocemente il contenuto, senza considerare i frame precedenti o successivi, come invece accade nelle codifiche iterframe come H.264/MPEG-4 AVC.\@
         \\
         Con l'accrescere della potenza di calcolo, M-JPEG  è stato soppiantato da MPEG-4, per la resa qualitativamente molto migliore a parità di bitrate, infatti JPEG è inefficente nello spazio di archiviazione. Inoltre M-JPEG non sfrutta tecniche di predizione spaziale, come H.264/MPEG-4 AVC, che permetterebbero un ulteriore risparmio di spazio, a discapito di una maggiore potenza di calcolo necessaria.

         \paragraph{MPEG-1}
         MPEG-1 è uno standard per la compressione con perdita di video e audio progettato per comprimere video con la qualità variabile tra il video di una \gls{VHS} e l'audio di un CD fino a 1.5Mbit/s senza peridta di qualità eccessive, permetendo la creazione di video CD, trasmissioni tv via satellite e via cavo e \gls{DAB}. Attualmente, MPEG-1 è il formato lossy con più supporto di compatibilità al mondo e uno degli standard che ha introdotto più noti è stato il formato audio MP3.



   \subsection{Protocolli}
      \subsubsection{Storia dei protocolli e standardizzazione}
         
      \subsubsection{Differenza tra protocolli real-time e on-demand}

      \subsubsection{Problemi dell'UDP con i firewall}

      \subsubsection{WebRTC}

   \subsection{Architettura di una piattaforma di streaming}
      \subsubsection{Struttura di massima}

      \subsubsection{Inbound}
         \paragraph{Streaming}

         \paragraph{Upload di file}

      \subsubsection{Storage}
         \paragraph{Sistemi elastici}

         \paragraph{Organizzazione dei contenuti}

      \subsubsection{Outbound}
         \paragraph{Standard e reti}

         \paragraph{CDN}

         \paragraph{Relay streaming}

         \paragraph{Codifica al volo}

   \subsection{Trasmissione dei contenuti in mobilità}
      \subsubsection{Difficoltà della trasmissione in mobilità}

      \subsubsection{Reti}
         \paragraph{Rete 4G}   
         \paragraph{WiFi}
         \paragraph{Bluetooth}

\section{Analisi dei requisiti}
   \subsection{Analisi preliminare}
      %Definizione dei requisiti
   
   \subsection{Definizione dei casi d'uso}
      %Diagramma UML
   
\section{Struttura della piattaforma}
   \subsection{Diagramma strutturale}
      %Diagramma UML
      %Spiegazione della funzione delle compnenti

   \subsection{Confronto con soluzioni esistenti}
      \subsubsection{YouTube}
      \subsubsection{Red5 Pro}
      \subsubsection{Contus Vplay}
      \subsubsection{Streamroot}
      \subsubsection{Wowza Streaming Engine}

\section{Tecnologie utilizzate}
   \subsection{Linguaggi --- Java}
      %Integrazione facile con Android
   
   \subsection{Piattaforma --- Tomcat8 EE}

   \subsection{Protocollo --- TCP su WebSocket}

   \subsection{Tipo di dati scambiati --- JSON vs XML}
      %Meno prolisso
      %Scelto per semplificare la codifica della piattaforma
      %Facile a comprendere
   \subsection{Strumetni di supporto}
      \subsubsection{Strumenti di versionamento}
         %Git e Bitbucket
      \subsubsection{Strumenti di test e verifica}
         %JUnit
      \subsubsection{Strumenti di analisi di rete}
         %Wireshark
      \subsubsection{Strumenti di continuous integration}
         %Docker
\section{Nuovi orizzonti}
   \subsection{Blockschain}
   \subsection{Streaming peer to peer}

\section{Qualifica}
   \subsection{Verifica}
      \subsubsection{Analisi statica}
      \subsubsection{Analisi dinamica}
   \subsection{Validazione}
      \subsubsection{Validazione interna}
      \subsubsection{Validazione esterna}