%**************************************************************
% Capitolo 2 - PERCHÉ - Scelta dello stage e rapporto con l'azienda
%**************************************************************
\chapter{Scelta dello stage e rapporto con l'azienda}
%TODO: introduzione capitolo

\section{Lo stage per l'azienda}
%TODO: sezione: rapporto gli stage dell'azienda

   \subsection{Necessità dell'azienda}

   %TODO: da rivedere
   \nomeAzienda{} ha deciso anche quest'anno di partecipare all'evento ``StageIt'', organizzato da Confindustria Padova in collaborazione con l'Università degli Studi di Padova, e proporre agli studenti uno tirocinio interno all'azienda. L'impresa è alla ricerca di nuovi neolaureati per arricchire il proprio team di sviluppatori, dato il recente aumento di clienti e la conseguente espansione. \nomeAzienda{} è, in particolare, interessata a studenti che sono appassionati di nuove tecnologie e che hanno interesse ad imparare nuove tecniche e a farne conoscere di nuove all'azienda stessa.

   \subsection{Risultati degli stage precedenti e seguito degli stagisti nell'azienda}
   %TODO: espandere la sezione
   L'azienda ha già organizzato tirocini con altri studenti e ha ottenuto risultati soddisfacenti. Molti degli stagisti sono tuttora all'interno dell'azienda, assunti con contratto a tempo determinato.


\section{Rapporto con il mio stage e l'azienda}
%TODO: sezione cosa mi interessa, cosa mi piace fare

   \subsection{Ambiti di interesse}
   Al momento della ricerca di uno stage ho prestato particolare attenzione alle aziende che proponevano un percorso legato ai miei interessi di studio. In particolare cercavo tirocini il cui argomento fosse compreso tra i seguenti:
   \begin{itemize}
      \item{Sicurezza;}
      \item{Sistemi ad alta concorrenza;}
      \item{Sistemi virtualizzati;}
      \item{Dispositivi IoT;}
      \item{DevOps.}
   \end{itemize}

   \subsection{Proposte di stage ricevute}
   

   \subsection{Scelta dello stage}

   \subsection{Scelta dell'azienda}

\section{Obiettivi dello stage}

   \subsection{Obiettibi obbligatori}

   \subsection{Obiettivi desiderabili}

   \subsection{Vincoli tecnologici}

\section{Pianificazione del lavoro}

   \subsection{Scelte sulla pianificazione}

   \subsection{Strumenti utilizzati}