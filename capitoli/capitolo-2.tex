%**************************************************************
% Capitolo 2 - PERCHÉ - Scelta dello stage e rapporto con l'azienda
%**************************************************************
\chapter{Scelta dello stage e rapporto con l'azienda}
%TODO: introduzione capitolo

\section{Lo stage per l'azienda}
%TODO: sezione: rapporto gli stage dell'azienda

   \subsection{Necessità dell'azienda}
   \nomeAzienda{} sta sviluppando un nuovo prodotto, un sistema di videoconferenza per assistenza remota da applicare ai lavori specialistici e all'interno di aziende manifatturiere. Il sistema è pensato per guidare un lavoratore inesperto, in situazioni difficili, da una persona con le conoscenze adeguate, per aiutarlo nelle sue mansioni. Per rendere l'esperienza pratica funzionale, l'azienda ha pensato di utilizzare dei visori per la realtà aumentata, concentrandosi particolarmente sui Google Glass. L'utente esperto dovrà essere in grado di vedere in tempo reale quello che l'altro vede, tramite una videocamera integrata negli occhiali, e di mostrargli dati ed indicazioni su quello che deve fare.
   \\
   \nomeAzienda{} ha già realizzato un prototipo di client Android del servizio e, dopo test di trasmissione interni al dispositivo, si sta preparando per il passaggio alla comunicazione su una rete locale, tramite una piattaforma di \engl{streaming} \engl{\gls{real-time}} proprietaria, ancora da sviluppare.
   %TODO: schema logico del sistema
   \\
   Il progetto del tirocinio proposto da \nomeAzienda{} consiste nella realizzazione del sistema di trasmissione di messaggi da utilizzare per la comunicazione tra i client del servizio.

   \subsection{Risultati degli stage precedenti e seguito degli stagisti nell'azienda}
   %TODO: espandere la sezione
   \nomeAzienda{} ha deciso anche quest'anno di partecipare all'evento ``StageIt'', organizzato da Confindustria Padova in collaborazione con l'Università degli Studi di Padova, e proporre agli studenti uno tirocinio interno all'azienda. L'impresa è alla ricerca di nuovi neolaureati per arricchire il proprio team di sviluppatori, dato il recente aumento di clienti e la conseguente espansione. \nomeAzienda{} è, in particolare, interessata a studenti che sono appassionati di nuove tecnologie e che hanno interesse ad imparare nuove tecniche e a farne conoscere di nuove all'azienda stessa.
   L'azienda ha già organizzato tirocini con altri studenti negli anni precedenti e ha ottenuto risultati soddisfacenti, tanto che molti degli ex stagisti sono stati assunti dall'azienda.

\section{Rapporto con il mio stage e l'azienda}
%TODO: sezione cosa mi interessa, cosa mi piace fare

   \subsection{Ambiti di interesse}
   Al momento della ricerca di uno stage ho prestato particolare attenzione alle aziende che proponevano un percorso legato ai miei interessi di studio. In particolare cercavo tirocini il cui argomento fosse compreso tra i seguenti:
   \begin{itemize}
      \item{Sicurezza;}
      \item{Sistemi ad alta concorrenza;}
      \item{Dispositivi IoT;}
      \item{Sistemi virtualizzati;}
      \item{Servizi cloud;}
      \item{DevOps;}
      \item{Sistemi multimediali.}
   \end{itemize}

   \subsection{Proposte di stage ricevute}
   Nei giorni immediatamente seguenti all'evento StageIt sono stato contattato da alcune aziende presenti. Tra le proposte che più mi parevano interessanti ho selezionato i progetti di IKS, Diana, Gsquared e \nomeAzienda{}.
   \\
   IKS proponeva un sistema di controllo di risorse e consumi di sistemi virtualizzati tramite \gls{Docker}: il software doveva fornire una chiara visione dello stato di ciascun servizio e riportare eventuali anomalie. Una volta presentatomi per un colloquio nella sede di Padova mi è stato proposto anche di aggregarmi ad un progetto sperimentale su \gls{AWS}, per testarne l'efficacia per possibili progetti futuri dell'azienda.
   \\
   Diana ha proposto un progetto per la realizzazione di un software i grado di unificare la grande mole di dati dell'azienda, sparsa tra i diversi sistemi aziendali e i \gls{CMS} dei loro clienti. Un altro progetto proposto, invece, prevedeva la realizzazione di un sistema per la gestione delle traduzioni dei testi dei prodotti, consentendo visualizzazione, modifica e la ricerca di testi ripetuti tramite Elasticsearch, per l'ottimizzazione delle spese di traduzione.
   \\
   Gsquared proponeva un progetto sperimentale per lo spostamento del proprio software di rendering \gls{CAD} per \gls{TAC} da un sistema locale ad uno client-server, con elaborazione dei dati su server virtualizzati. Il video elaborato viene poi servito ad un \gls{thin client}; alleggerendolo del carico di computazione del render.
   
   \subsection{Scelta dello stage}
   %TODO: espandere sezione
   Al momento della scelta di quale stage accettare, tra quelli proposti, ho preferito optare per lo stage che proponeva argomenti per me più nuovi e meno conosciuti.

   \subsection{Scelta dell'azienda}
   Durante la scelta dello stage ho considerato poco l'aspetto del futuro in azienda, dato che la mia scelta di continuare a studiare per la laurea magistrale è incompatibile con un lavoro a tempo pieno. Il mio interesse più grande era quello di poter collaborare con persone più esperte di me per imparare cose nuove.


\section{Obiettivi dello stage}

   \subsection{Obiettibi obbligatori}

   \subsection{Obiettivi desiderabili}

   \subsection{Vincoli tecnologici}

\section{Pianificazione del lavoro}

   \subsection{Scelte sulla pianificazione}

   \subsection{Strumenti utilizzati}