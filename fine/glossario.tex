%**************************************************************
% Glossario
%**************************************************************
\newglossaryentry{AWS}
{
    name={AWS},
    description={Amazon Web Services (AWS) è una collezione di servizi di \engl{cloud} computing \engl{on-demand} offerta da Amazon},
    first={Amazon Web Services (AWS)},
    long={Amazon Web Services}
}

\newglossaryentry{VPS}
{
    name={VPS},
    description={Un Virtual Private Server (VPS) è un'istanza di un sistema che viene eseguito in un ambiente virtuale},
    first={Virtual Private Server (VPS)},
    long={Virtual Private Server}
}

\newglossaryentry{IoT}
{
    name={IoT},
    description={Per Internet of Things (IoT) ci si riferisce all'estensione di Internet agli oggetti comuni, che diventano intelligenti e comunicano dati su sé stessi e sul mondo che li circonda e allo stesso tempo accedere ad informazioni altrove nella rete},
    first={IoT},
    long={Internet of Things}
}

\newglossaryentry{wearable}
{
    name={Wearable},
    text={wearable},
    description={Si dice \engl{wearable} un dispositivo elettronico indossabile on impiantabile. In generale questi dispositivi offrono delle funzionalità di notifica legate agli smartphone oppure contengono sensori per la rilevazione di attività fisica e sono un esempio di dispositivo \gls{IoT}}
}

\newglossaryentry{SEO}
{
    name={SEO},
    description={Si definisce Search Engine Optimization (SEO) l'attività di ottimizzazione dei contenuti di una pagina web per l'indicizzazione da parte dei motori di ricerca},
    first={SEO},
    long={Search Engine Optimization}
}

\newglossaryentry{CVS}
{
    name={CVS},
    description={É detto Concurrent Versioning System (CVS) un \engl{software} che implementa un sistema di controllo di versione. Il sistema mantiene organizzati i cambiamenti fatti a un certo numero di file e permette a molti sviluppatori di collaborare accedendo alle stesse risorse},
    first={CVS},
    long={Concurrent Versioning System}
}

\newglossaryentry{ubiquitous computing}
{
    name={Ubiquitous computing},
    description={L'ubiquitous computing è un nuovo modello di interfaccia uomo macchina, secondo il quale ogni persona, nelle sue azioni quotidiane, può entrare in contatto con un enorme numero di dispositivi elettronici, più o meno specializzati, che comunicano tra loro e possono collaborare a uno scopo. Si differenzia dal precedente modello uomo-macchina per la completa integrazione dell'elaborazione delle informazioni all'interno del singolo dispositivo, senza dipendere da un nodo computazionale esterno},
    text={ubiquitous computing}
}

\newglossaryentry{project manager}
{
    name={Project manager},
    description={Il \engl{project manager} di un progetto è il responsabile dell'organizzazione dei processi e della loro pianificazione all'interno di esso},
    text={project manager}
}

\newglossaryentry{real-time}
{
    name={Real-time},
    description={In informatica un sistema \engl{real-time} è un sistema in cui la correttezza del risultato delle sue computazioni non solo dipende della correttezza logica, ma anche dal tempo impiegato per raggiungerlo. Un sistema \engl{real-time} deve poter garantire che il tempo di computazione non superi un certo limite superiore, deciso in progettazione},
    text={real-time}
}

\newglossaryentry{Docker}
{
    name={Docker},
    description={Docker è un \engl{software} \engl{open-source} per la virtualizzazione di sistemi operativi in ``container'' isolati e controllati. Il metodo utilizzato da Docker sfrutta il sistema di isolamento delle risorse del \engl{kernel} Linux, permettendo la coesistenza di più container sulla stessa macchina e limitando gli sprechi di risorse collegati all'utilizzo di una macchina virtuale completa}
}

\newglossaryentry{CMS}{
    name={Content Management System (CMS)},
    text={CMS},
    first={Content Management System (CMS)},
    description={Un Content Management System (CMS) è un \engl{software} di supporto alla creazione, modifica e gestione di contenuti digitali}
}

\newglossaryentry{CAD}{
    name={Computer-Aided Design (CAD)},
    text={CAD},
    description={Computer-Aided Design (CAD) indica un \engl{software} volto all'utilizzo di tecnologie per la computer grafica per supportare l'attività di progettazione di modelli, soprattutto 3D}
}

\newglossaryentry{TAC}{
    name={Tomografia Assiale Computerizzata (TAC)},
    text={TAC},
    description={In medicina la Tomografia Assiale Computerizzata (TAC) è una metodica diagnostica per immagini che consente di riprodurre sezioni o strati ed effettuare elaborazioni tridimensionali dei dati ottenuti}
}

\newglossaryentry{thin client}{
    name={Thin client},
    text={\engl{thin client}},
    description={Un \engl{thin client} è un \engl{client} leggero pensato per connettersi a un server remoto che esegue tutte le operazioni sensibili. Si contrappone al convenzionale \engl{fat client} nel quale è il \engl{client} stesso ad eseguire la maggior parte delle operazioni e può comunicare parte dei dati ad altri dispositivi}
}

\newglossaryentry{Gantt}
{
    name={Gantt},
    description={Un diagramma di Gantt è un diagramma a barre pensato per mostrare su una scala temporale le attività di un processo, le risorse che occupano, il tempo impiegato e le dipendenze di ciascuna. Questo diagramma è molto utile per stimare i tempi di sviluppo di un prodotto e fissare milestone e scadenze adeguate}
}

\newglossaryentry{PERT}
{
    name={Program Evaluation and Review Technique (PERT)},
    description={Program Evaluation and Review Technique (PERT) è uno strumento utilizzato per la gestione di un progetto pensato per analizzarne e rappresentarne i task necessari al suo completamento},
    text={PERT}
}

\newglossaryentry{VHS}
{
    name={Video Home System (VHS)},
    text={VHS},
    description={Video Home System è un sistema di registrazione video in formato analogico su supporto magnetico}
}

\newglossaryentry{DAB}
{
    name={Digital Audio Broadcasting (DAB)},
    text={DAB},
    description={Il Digital Audio Broadcasting (DAB) è uno \engl{standard} di radiotrasmissione digitale che permette la trasmissione di un segnale audio, in genere programmi radiofonici, con una qualità paragonabile a quella di un \gls{CD} audio; supporta il multiplexing, la trasmissione di più tracce contemporaneamente sullo stesso canale e utilizza l'algoritmo di compressione HE-AAC, che prevede resistenza ai disturbi e un'ottima efficienza}
}

\newglossaryentry{CDN}
{
    name={Content Delivery Network (CDN)},
    text={CDN},
    description={Una Content Delivery Network (CDN) è una rete di computer collegati tra di loro tramite una rete Internet, che collaborano in maniera trasparente per distribuire contenuti; in genere \engl{streaming} audio e video}
}

\newglossaryentry{ISP}
{
    name={Internet Service Provider (ISP)},
    text={ISP},
    first={Internet Service Provider (ISP)},
    description={Un Internet Service Provider (ISP) è una struttura commerciale o un'organizzazione che fornisce l'accesso a servizi Internet, dietro la stipulazione di un contratto}
}

\newacronym{JVM}{JVM}{Java Virtual Machine}

\newglossaryentry{API}
{
    name={Application Programming Interface (API)},
    text={API},
    description={In informatica con API si indica ogni insieme di procedure disponibili al programmatore, di solito raggruppate a formare un set
    di strumenti specifici per l’espletamento di un determinato compito all’interno
    di un certo programma. La finalità è ottenere un’astrazione, in genere tra
    l’\engl{hardware} e il programmatore o tra \engl{software} a basso e quello ad alto livello,
    semplificando, così, le attività di programmazione}
}

\newacronym{HTML}{HTML}{Hypertext Markup Language}

\newacronym{XML}{XML}{Extensible Markup Language}

\newacronym{JSON}{JSON}{JavaScript Object Notation}

\newacronym{EC2}{EC2}{Elastic Compute Cloud}

\newacronym{SDK}{SDK}{Software Development Kit}

\newacronym{S3}{S3}{Simple Storage Service}

\newacronym{RTSP}{RTSP}{Real Time Streaming Protocol}

\newacronym{RTMPT}{RTMPT}{Real Time Messaging Protocol over HTTP}

\newacronym{RTMP}{RTMP}{Real Time Messaging Protocol}

\newacronym{HLS}{HLS}{\acrshort{HTTP} Live Streaming}

\newacronym{DRM}{DRM}{Digital Rights Management}

\newacronym{UML}{UML}{Unified Modeling Language}

\newacronym{IEEE}{IEEE}{Institute of Electrical and Electronics Engineers}

\newacronym{LTE}{LTE}{Long Term Evolution}

\newacronym{HTTP}{HTTP}{HyperText Transfer Protocol}

\newacronym{OSI}{OSI}{Open Systems Interconnection}

\newacronym{RRCP}{RRCP}{Realtek Remote Control Protocol}

\newacronym{RTP}{RTP}{Real-time Transport Protocol}

\newacronym{RTMCP}{RTMCP}{Real Time Message Control Protocol}

\newacronym{NAT}{NAT}{Network Address Translation}

\newacronym{TLS}{TLS}{Transport Layer Security}

\newacronym{BSD}{BSD}{Berkeley Software Distribution}

\newacronym{GIF}{GIF}{Graphics Interchange Format}

\newacronym{MPEG}{MPEG}{Moving Picture Experts Group}

\newacronym{JPEG}{JPEG}{Joint Photographic Experts Group}

\newacronym{CD}{CD}{Compact Disc}

\newacronym{PNG}{PNG}{Portable Network Graphics}

\newglossaryentry{trasformata DCT}{
    name={Trasformata Discreta del Coseno},
    text={trasformata discreta del coseno},
    description={La trasformata discreta del coseno (\gls{DCT}) è una funzione che provvede alla compressione spaziale, capace di rilevare variazioni tra un area e quella contigua}
}

\newacronym{DCT}{DCT}{Discrete Cosine Transform}

\newacronym{TCP}{TCP}{Transmission Control Protocol}

\newacronym{UDP}{UDP}{User Datagram Protocol}

\newacronym{SCTP}{SCTP}{Stream Control Transmission Protocol}

\newacronym{MPEG-DASH}{MPEG-DASH}{Dynamic Adaptive Streaming over \acrshort{HTTP}}

\newglossaryentry{transcodificatore}{
    name={Transcodificatore},
    text={transcodificatore},
    plural={transcodificatori},
    description={Descrizione TRANSCODIFICATORE} %TODO TRANSCODIFICATORE
}