%**************************************************************
% Glossario
%**************************************************************
\newglossaryentry{AWS}
{
    name={AWS},
    description={Amazon Web Services (AWS) è una collezione di servizi di cloud computing on demand offerta da Amazon},
    first={Amazon Web Services (AWS)},
    long={Amazon Web Services}
}

\newglossaryentry{VPS}
{
    name={VPS},
    description={Un Virtual Private Server (VPS) è un'istanza di un sistema che viene eseguito in un ambiente virtuale},
    first={Virtual Private Server (VPS)},
    long={Virtual Private Server}
}

\newglossaryentry{IoT}
{
    name={IoT},
    description={Per Internet of Things (IoT) ci si riferisce all'estensione di Internet agli oggetti comuni, che diventano intelligenti e comunicano dati su se stessi e sul mondo che li circonda e allo stesso tempo accedere ad informazioni altrove nella rete},
    first={IoT},
    long={Internet of Things}
}

\newglossaryentry{wearable}
{
    name={Wearable},
    text={wearable},
    description={Si dice wearable un dispositivo elettronico indossabile on impiantabile. In generale questi dispositivi offrono delle funzionalità di notifica legate agli smartphone oppure contengono sensori per la rilevazione di attività fisica e sono un esempio di dispositivo \gls{IoT}}
}

\newglossaryentry{SEO}
{
    name={SEO},
    description={Si definisce Search Engine Optimization (SEO) l'attività di ottimizzazione dei contenuti di una pagina web per l'indicizzazione da parte dei motori di ricerca},
    first={SEO},
    long={Search Engine Optimization}
}

\newglossaryentry{CVS}
{
    name={CVS},
    description={É detto Concurrent Versioning System (CVS) un software che implementa un sistema di controllo di versione. Il sistema mantiene organizzati i cambiamenti fatti ad un certo numero di file e permette a molti sviluppatori di collaborare accedendo alle stesse risorse},
    first={CVS},
    long={Concurrent Versioning System}
}

\newglossaryentry{ubiquitous computing}
{
    name={Ubiquitous computing},
    description={L'ubiquitous computing è un nuovo modello di interfaccia uomo macchina, secondo il quale ogni persona, nelle sue azioni quotidiane, può entrare in contatto con un enorme numero di dispositivi elettronici, più o meno specializzati, che comunicano tra loro e possono collaborare ad uno scopo. Si differenzia dal precedente modello uomo-macchina per la completa integrazione dell'elaborazione delle informazioni all'interno del singolo dispositivo, senza dipendere da un nodo computazionale esterno}
}

\newglossaryentry{project manager}
{
    name={Project manager},
    description={Il \engl{project manager} di un progetto è il rsponsabile dell'organizzazione dei processi e della loro pianificazione all'interno di esso},
    text={project manager}
}

\newglossaryentry{real-time}
{
    name={Real-time},
    description={In informatica un sistema real-time è un sistema in cui la correttezza del risultato delle sue computazioni non solo dipende della correttezza logica, ma anche dal tempo impiegato per raggiungerlo. Un sistema real-time deve poter garantire che il tempo di computazione non superi un certo limite superiore, deciso in progettazione},
    text={real-time}
}

\newglossaryentry{Docker}
{
    name={Docker},
    description={Docker è un software open-source per la virtualizzazione di sistemi operativi in ``container'' isolati e controllati. Il metodo utilizzato da Docker sfrutta il sistema di isolamento delle risorse del kernel Linux, permettendo la coesistenza di più container sulla stessa macchina e limitando gli sprechi di risorse collegati all'utilizzo di una macchina virtuale completa}
}

\newglossaryentry{CMS}{
    name={Content Management System (CMS)},
    text={CMS},
    first={Content Management System (CMS)},
    description={Un Content Management System (CMS) è un software di supporto alla creazione, modifica e gestione di contenuti digitali.}
}

\newglossaryentry{CAD}{
    name={Computer-Aided Design (CAD)},
    text={CAD},
    description={Computer-Aided Design (CAD) indica un software volto all'utilizzo di tecnologie per la computer grafica per supportare l'attività di progettazione di modelli, soprattutto 3D}
}

\newglossaryentry{TAC}{
    name={Tomografia Assiale Computerizzata (TAC)},
    text={TAC},
    first={Tomografia Assiale Computerizzata (TAC)},
    description={In medicina la tomografia assiae computerizzata (TAC) è una metodica diagnostica per immagini che consente di riprodurre sezioni o strati ed effettuare elaborazioni tridimensionali dei dati ottenuti}
}

\newglossaryentry{thin client}{
    name={Thin client},
    text={thin client},
    description={Un \engl{thin client} è un \engl{client} leggero pensato per connettersi ad un server remoto che esegue tuttle le operazioni sensibili. Si contrappone al convenzionale \engl{fat client} nel quale è il client stesso ad eseguire la maggior parte delle operazioni e può comunicare parte dei dati ad altri dispositivi}
}

\newglossaryentry{Gantt}
{
    name={Gantt},
    description={Un diagamma di Gantt è un diagramma a barre pensato per mostrare su una scala temporale le attività di un processo, le risorse che occupano, il tempo impiegato e le dipendenze di ciascuna. Questo diagramma è molto utile per stimare i tempi di sviluppo di un prodotto e fissare milestone e scadenze adeguate}
}

\newglossaryentry{PERT}
{
    name={Program evaluation and review technique (PERT)},
    description={Program evaluation and review technique (PERT). è uno strumetno utilizzato per la gestione di un progetto pensato per analizzare e rappresentare i task necessari al completamento di un progetto},
    text={PERT}
}