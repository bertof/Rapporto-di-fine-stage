%**************************************************************
% Glossario
%**************************************************************
\newglossaryentry{AWS}
{
    name={AWS},
    description={Amazon Web Services (AWS) è una collezione di servizi di cloud computing on demand offerta da Amazon},
    first={Amazon Web Services (AWS)},
    long={Amazon Web Services}
}

\newglossaryentry{VPS}
{
    name={VPS},
    description={Un Virtual Private Server (VPS) è un'istanza di un sistema che viene eseguito in un ambiente virtuale},
    first={Virtual Private Server (VPS)},
    long={Virtual Private Server}
}

\newglossaryentry{IoT}
{
    name={IoT},
    description={Per Internet of Things (IoT) ci si riferisce all'estensione di Internet agli oggetti comuni, che diventano intelligenti e comunicano dati su se stessi e sul mondo che li circonda e allo stesso tempo accedere ad informazioni altrove nella rete},
    first={IoT},
    long={Internet of Things}
}

\newglossaryentry{wearable}
{
    name={Wearable},
    text={wearable},
    description={Si dice wearable un dispositivo elettronico indossabile on impiantabile. In generale questi dispositivi offrono delle funzionalità di notifica legate agli smartphone oppure contengono sensori per la rilevazione di attività fisica e sono un esempio di dispositivo \gls{IoT}}
}

\newglossaryentry{SEO}
{
    name={SEO},
    description={Si definisce Search Engine Optimization (SEO) l'attività di ottimizzazione dei contenuti di una pagina web per l'indicizzazione da parte dei motori di ricerca},
    first={SEO},
    long={Search Engine Optimization}
}

\newglossaryentry{CVS}
{
    name={CVS},
    description={É detto Concurrent Versioning System (CVS) un software che implementa un sistema di controllo di versione. Il sistema mantiene organizzati i cambiamenti fatti ad un certo numero di file e permette a molti sviluppatori di collaborare accedendo alle stesse risorse},
    first={CVS},
    long={Concurrent Versioning System}
}

\newglossaryentry{ubiquitous computing}
{
    name={Ubiquitous computing},
    description={L'ubiquitous computing è un nuovo modello di interfaccia uomo macchina, secondo il quale ogni persona, nelle sue azioni quotidiane, può entrare in contatto con un enorme numero di dispositivi elettronici, più o meno specializzati, che comunicano tra loro e possono collaborare ad uno scopo. Si differenzia dal precedente modello uomo-macchina per la completa integrazione dell'elaborazione delle informazioni all'interno del singolo dispositivo, senza dipendere da un nodo computazionale esterno}
}

\newglossaryentry{project manager}
{
    name={Project manager},
    description={Il \engl{project manager} di un progetto è il rsponsabile dell'organizzazione dei processi e della loro pianificazione all'interno di esso},
    text={project manager}
}